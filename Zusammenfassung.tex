\documentclass[12pt]{article}
\usepackage{amsfonts}
\usepackage{fancyhdr}
\usepackage[a4paper, top=2.5cm, bottom=2.5cm, left=2.2cm, right=2.2cm]{geometry}
\usepackage{times}
\usepackage{amsmath}
\usepackage{changepage}
\usepackage{amssymb}
\usepackage{graphicx}%
\setcounter{MaxMatrixCols}{30}
\newtheorem{theorem}{Theorem}
\newtheorem{acknowledgement}[theorem]{Acknowledgement}
\newtheorem{algorithm}[theorem]{Algorithm}
\newtheorem{axiom}{Axiom}
\newtheorem{case}[theorem]{Case}
\newtheorem{claim}[theorem]{Claim}
\newtheorem{conclusion}[theorem]{Conclusion}
\newtheorem{condition}[theorem]{Condition}
\newtheorem{conjecture}[theorem]{Conjecture}
\newtheorem{corollary}[theorem]{Corollary}
\newtheorem{criterion}[theorem]{Criterion}
\newtheorem{definition}[theorem]{Definition}
\newtheorem{example}[theorem]{Example}
\newtheorem{exercise}[theorem]{Exercise}
\newtheorem{lemma}[theorem]{Lemma}
\newtheorem{notation}[theorem]{Notation}
\newtheorem{problem}[theorem]{Problem}
\newtheorem{proposition}[theorem]{Proposition}
\newtheorem{remark}[theorem]{Remark}
\newtheorem{solution}[theorem]{Solution}
\newtheorem{summary}[theorem]{Summary}
\usepackage{enumitem}
\usepackage[utf8]{inputenc}
\newenvironment{proof}[1][Proof]{\textbf{#1.} }{\ \rule{0.5em}{0.5em}}
\usepackage{tikz}
\usepackage{graphicx}
\usepackage{wrapfig}
\usepackage{float}
\usepackage{datetime}
\usepackage{hyperref}
\newdateformat{specialdate}{\twodigit{\THEDAY}.\twodigit{\THEMONTH}.\THEYEAR}

\newcommand{\red}[1]{\color{red}#1\color{black}}

\begin{document}
	
	\title{Zusammenfassung - Investition und Finanzierung}
	\author{Timo Bergerbusch 344408}
	\date{\specialdate\today}
	\maketitle
	\section{Entscheidungsregeln}
	\begin{enumerate}
		\item Grenzrate der Substitution $GRS=GRT$ Grenzrate der Transformation
			\begin{itemize}
				\item Modellwelt Thema 1
				\item benötigt differenzierbare Transformationsfunktion
				\item Zwei-Zeitpunkt-Betrachtung
			\end{itemize}
		\item Grenzrendite = Kapitalmarktzins\label{2Investitionsregel}
			\begin{itemize}
				\item benötigt differenzierbare Transformationsfunktion
				\item Zwei-Zeitpunkt-Betrachtung
			\end{itemize}
		\item Maximierung des Kapitalwertes 
			\begin{itemize}
				\item Auch für Mehr-Perioden-Fall anwendbar
				\item auch für nicht bel. teilbare Projekte
				\item echt besser als die \hyperref[2Investitionsregel]{2. Investitionsregel}
			\end{itemize}
	\end{enumerate}
	
	\section{Fachbegriffe}
	\begin{itemize}
		\item Investitionsertragsfunktion $F(I)$: Herleitung wie in \hyperref[Thema1Aufgabe1]{Thema 1 Aufgabe 1} \\
			Verfügt über 3 Eigenschaften: \begin{enumerate}
				\item $F(0)=0$
				\item positiver Ertrag: $F'(I) >0$ für $I>0$
				\item abnehmender Grenznutzen (degressiv): $F''(I) <0$ für $I>0$
			\end{enumerate}
		\item Rendite: $\frac{\text{Ertrag}\cdot100}{\text{Investition}}-1$\label{Rendite}
		\item Transformationsfunktion:  $C_1=F(W_0-C_0)$
		\item optimales Investitionsvolumen: \\
			\textbf{Kriterium}: Steigung Transformationskurve $-F'(I) = -(1+i)$ Steigung Kapitalmarkt-gerade\\
			$\max U(C_0;C_1)$ unter der NB $C_1=F(W_0-C_0)$
			\begin{enumerate}
				\item einsetzen von $C_1$
				\item ableiten mittels $\frac{\partial U}{\partial C_0}$
				\item lösen nach $C_0$
			\end{enumerate}
		\item Indifferenzkurve:
			\begin{itemize}
				\item[Def.:] eine Kurve im $(C_0;C_1)$-Diagramm für die ein Entscheider keinen unterschied zwischen dem $C_0$ Konsum jetzt oder dem $C_1$ Konsum in $t_1$ macht
				\item bei geg. Nutzenfunktion $\overline{U}=C_0^x\cdot C_1^y \Leftrightarrow C_1 = \overline{U}^{\frac{1}{y}}\cdot C_0^{-\frac{x}{y}} $
			\end{itemize}
		\item Kapitalwert $\kappa$: 
			\begin{itemize}
				\item für Zeiträume $t_o, \dots, t_n$ und Zinssatz $i$ gilt $\kappa=\sum_{j=0}^n \frac{t_j}{(1+i)^j}$ \label{Kapitalwert}
				\item bei geg. $RBF(i;T)$: $\kappa=RBF(i;T)\cdot z -A_0$, gleichbleibende Einzahlung $z$, Anfangsauszahlung $A_0$
			\end{itemize}
		
		\item Differenzinvestition: $\kappa^{A-B}=\kappa_A - \kappa_B$
		\item Zahlungsreihe einer Differenzinvestition (A-B): Für alle $t\in T$ berechne $z_t^{(A)}-z_t^{(B)}$ und anschließend den \hyperref[Kapitalwert]{Kapitalwert} \label{ZahlungsreiheEinerDifferenzinvestition}
		\item Rentenbarwertaktor $RBF$ \label{Rentenbarwertfaktor}:
			\begin{itemize}
				\item[Def.:] Der RBF entspricht dem Kaptialwert einer gleichbleibenden Einzahlung von genau 1 GE in den Zeitpunkten $t=1$ bis $t=T$
				\item $RBF(i;T)=\frac{(1+i)^T-1}{(1+i)^T\cdot i}$ für Zeitraum $t$ bis $T$ und Zinsfuß $i$ 
			\end{itemize}
		\item Annuität:\label{Annuitaet}
			\begin{itemize}
				\item[Def.:] Welche gleichbleibende Einzahlung von t=1 bis t=T bei einem Kalkulationszinsfuß $i$ erforderlich ist um einen Kapitalwert $\kappa$ von genau 1 GE zu generieren.
				\item Annuitätsfaktor = $ANN(i;T)=\frac{1}{RBF(i;T)}$ 
				\item Berechnung des konst. Zahlungsüberschusses pro Periode/ Annuität: $z=\frac{\kappa + A_0}{RBF(i;T)}$
			\end{itemize}
		\item Ertragswert $\eta_0$:
			\begin{itemize}
				\item[Def.:] Ertragswert $\eta_0$ entspricht dem Kapitalwert der Einzahlungsüberschüsse
				\item $\eta_0=\kappa-A_0 = RBF(i;T)\cdot z $
			\end{itemize}
	\end{itemize}
	\section{Dynexite Aufgaben}
	\subsection*{Thema 1}
%	\begin{enumerate}
%		\setlength{\itemindent}{1cm}
		\subsubsection*{Aufgabe 1}
		\label{Thema1Aufgabe1} Vorgehen:
		\begin{enumerate}
			\setlength{\itemindent}{1cm}
			\item berechne \hyperref[Rendite]{Renditen}$R_1,\dots, R_n$
			\item sortiere absteigend nach Renditen: $(I_{max},I_{2max},\dots,I_{least})$
			\item füge ein: $(R_{max}/100+1)*I$ für $0 < I \le \text{Investitionsvolumen}_{max}$
			\item füge ein: $E_max + (R_{2max}/100+1)\cdot I$ für \\ $\text{Investitionsvolumen}_{max} < I \le \text{Investitionsvolumen}_{max}+\text{Investitionsvolumen}_{2max}$, etc 
		\end{enumerate}
		\subsubsection*{Aufgabe 2}
		Ang. es schließen sich Projekt X und Y aus und $R_X>R_Y$. Vorgehen:
		\begin{enumerate}
			\setlength{\itemindent}{1cm}
			\item Analog zu \hyperref[Thema1Aufgabe1]{Aufgabe 1} Punkte 1 und 2
			\item Führe die Programme mit max. Renditen durch.
			\item[\red{!Wichtig!}] wenn $I_Y>I_X$: berechne $\frac{E_X}{R_Y/100+1} =x$.
				\begin{itemize}
					\setlength{\itemindent}{2cm}
					\item[$x>I_Y$:] nicht zu ändern
					\item[sonst:] füge Zeile hinzu mit: ${\sum_{i\in\text{durchgef. Proj.}}E_i} + (\frac{I_Y}{100}+1)*(I-\sum_{i\in\text{durchgef. Proj.}}I_i)$
				\end{itemize}
		\end{enumerate}
	\subsection*{Thema 2}
		\subsubsection*{Aufgabe 1}\label{Thema2Aufgabe1}
		Berechne die Kaptialwerte $\kappa_0, \dots, \kappa_n$ mit \hyperref[Kapitalwert]{dieser} Formel.
		\begin{itemize}
			\setlength{\itemindent}{1cm}
			\item[$\kappa_i>0$] durchführen
			\item[$\kappa_i =0$] indifferent
			\item[$\kappa_i<0$] nicht durchführen
		\end{itemize}
		Bei ausschließenden Projekten führe dasjenige aus, welches den höheren Kapitalwert hat und führe dies durch g.d.w. dessen $\kappa>0$
		\subsubsection*{Aufgabe 2}\label{Thema2Aufgabe2}
		Erster Teil analog zu \hyperref[Thema2Aufgabe1]{Thema 2 Aufgabe 1}. \\
		Berechne danach die \hyperref[ZahlungsreiheEinerDifferenzinvestition]{Zahlungsreihe einer geg. Differenzinvestition}
		\subsubsection*{Aufgabe 3}
		100\% analog zu \hyperref[Thema2Aufgabe2]{Thema 2 Aufgabe 2}. Unterschiede sind nur die Werte ($> 1000$).
	\subsection*{Thema 3}
		\subsubsection*{Aufgabe 1}
		\begin{enumerate}
			\item Berechne den $RBF(i;T)=\frac{(1+i)^T-1}{(1+i)^T*i}$, den Kapitalwert $\kappa=RBF(i;T)\cdot z -A_0$ und Ertragswert $\eta = \kappa + A_0$ ($z$ ist die gleichmäßige Entnahme)
			\item Berechne den Annuitätenfaktor $ANN(i;T)=\frac{1}{RBF(i;T)}$ und Annuität $\frac{\eta-A_0}{RBF(i;T)} = \kappa\cdot ANN(i;T)$
			\item Berechne die Annuitäten erneut nur nimm diesmal $ANN(i;T)$ des am längsten laufenden Projektes
		\end{enumerate}
		\subsubsection*{Aufgabe 2}
		Gegeben $i, T, \kappa, \eta$
		\begin{enumerate}[label=\alph*)]
			\item Rechne $\kappa \cdot ANN(i;T)$
			\item Rechne $ANN(i;T)$
			\item Rechne $\eta*RBF(i;T)$
		\end{enumerate}
		\subsubsection*{Aufgabe 3} \label{Thema3Aufgabe3}
		\begin{itemize}
			\item[a)]
			\item Abschreibung: bei linearer Abschreibung $D_t=A_0/T$
			\item durchschnittliche Mittelbindung = $(2\cdot MB_{t-1}-D_t)/2$ \\
			\textbf{Tipp}: fange bei $t=T$ an mit $\emptyset\text{-MB}_{t}=\frac{1}{2}D_t$ und rechne für $t=t-1$ einfach $\emptyset\text{-MB}_{t}=\emptyset\text{-MB}_{t-1}+D_t$
			\item kalkulatorische Zinsen $kZ_t=i\cdot \emptyset\text{-MB}_t$
			\item Periodengewinn $G_t=x_t*(p_t-k_{v,t})-k_{f,t}-D_t-kZ_t$
			\item durchschnittlicher Periodengewinn $G_t^\prime=\frac{\sum_{t=1}^T G_t}{T}$
			\item[b)]
			\item $z_0=-A_0$ und $z_t=x_t*(p_t-k_{v,t})-k_{f,t}$
			\item \hyperref[Kapitalwert]{Kapitalwert $\kappa$}, \hyperref[Annuitaet]{test $ANN(i;T)$} und \hyperref[Annuitaet]{Annuität} wie gewohnt
		\end{itemize}
	
		\subsubsection*{Aufgabe 4}
		\begin{enumerate}[label=\alph*)]
			\item Gegeben sind eine Relation zwischen $K_{v,t}$ und $p_t$. Nutze diese um zwei Gleichungen mit Parametern $a$ und $b$ zu erhalten und löse das Gleichungssystem.
			\item Nutze die Formel aus $a)$ für die ersten paar Lücken und berechne den Rest wie in \hyperref[Thema3Aufgabe3]{Thema 3 Aufgabe 3}
			\item ebenfalls Analog zu \hyperref[Thema3Aufgabe3]{Thema 3 Aufgabe 3}
		\end{enumerate}
\end{document}


















